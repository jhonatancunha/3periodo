\documentclass[a4paper, 12pt]{article}
%%%%%%%%%%% Pacotes utilizados
\usepackage[brazil]{babel}
\usepackage[utf8]{inputenc}
\usepackage{verbatim}
\usepackage[normalem]{ulem} %para 
\usepackage{indentfirst}
\usepackage{setspace}
\usepackage{tikz}
\usetikzlibrary{shapes,arrows, arrows.meta}

\usepackage{float}

\usepackage[backend=biber, style=numeric]{biblatex}

 \addbibresource{referencia.bib}


%%%%%%%%%%%%%%% Configurações
\setlength{\textwidth}{16cm}
\setlength{\textheight}{23cm}
\setlength{\evensidemargin}{-1cm} \setlength{\oddsidemargin}{0.5cm}
\setlength{\topmargin}{0cm}

\usepackage{fancyhdr}

\pagestyle{fancy}
\fancyhf{}
\lhead{\textbf{Nome:} Jhonatan Guilherme de Oliveira Cunha}
\rhead{\textbf{RA:} 2135590}
\cfoot{\thepage}

\hoffset= -0.4cm
\voffset=-0.9cm


\tikzstyle{class}=[
	rectangle,
	draw=black,
	text centered,
	anchor=north,
	text=black
]
\tikzstyle{myarrow}=[->, diamond, thick]

\tikzset{UMLRectangle/.style={class ,rectangle, text width=7em, text centered}}


%%%%%%%%%%%%% Início do documento


\begin{document}
	
	\hspace{5cm}
	
	\begin{large}
		\begin{center}
			\textbf{UNIVERSIDADE TECNOLÓGICA FEDERAL DO PARANÁ}\newline
			\textbf{CAMPUS CAMPO MOURÃO}
		\end{center}
	\end{large}
	
	\hspace{20cm}
	
	\begin{center}
		\textbf{RESENHA CAPÍTULO 1 - LIVRO ``UML Essencial
		FOWLER, Martin''}
	\end{center}

	\hspace{10cm}
	
	\section{O que é a UML?}
	
	\onehalfspacing
	\textbf{UML} é uma família de notações gráficas, sua sigla vem de \textit{Unified Modeling Language} (Linguagem de Modelagem Unificada) \cite{umlEssencial}. Este tipo de linguagem existe há muito tempo na industria de software, uma de suas principais funções é a ajuda na descrição e no projeto de sistemas de software, especificamente no paradigma orientado a objetos (OO).
	
	\section{Maneiras de usar a UML}
	
	Existem varias maneiras diferentes de utilizar a \textbf{UML} como ferramenta, de acordo com \textbf{Steve Mellor} e \textbf{Martin Fowler} três modos são principais. Sendo eles: esboço, projeto e linguagem de programação \cite{umlEssencial}.

	
	\subsection{Esboço}
	
		È principalmente usado para transmitir informações sobre um sistema, sendo elas diferentes em cada objetivo que deseja atingir. A fim de exemplificar, diferenciaremos entre desenvolvimento e engenharia reversa:
		
		\begin{itemize}
			\item \textbf{Desenvolvimento:} destacamos ideias e alternativas que utilizaremos no código, ou seja, visualizamos tudo antes de começar a programar.
			
			\item \textbf{Engenharia Reversa:} utilizamos para explicar o funcionamento de uma parte do sistema.
		\end{itemize}
	
	\subsection{Projeto}
	
		Sua principal característica está em sempre construir um modelo completo, ou seja, um projeto detalhado para o programador codificar, possibilitando-o a seguir o desenvolvimento como uma tarefa simples e direta. 
	
	
	\subsection{Linguagem de Programação}
	
		Neste caso programadores utilizam a \textbf{UML} como linguagem de programação, ou seja, os diagramas são compilados diretamente para código-fonte, utilizando ferramentas sofisticadas (MDA e UML) \cite{umlEssencial}.
		
		 A partir da \textbf{UML 2} podemos utilizar três tipos de diagramas a fim de modelar lógicas comportamentais, sendo eles: diagramas de interação, estado e atividade.
	
	\subsection{Perspectivas de Modelagem}
	
		Podemos enxergar a \textbf{UML} em duas variações de modelagem, sendo elas: conceitual e modelagem de software.
		
		\begin{itemize}
			\item \textbf{Conceitual:} representa uma descrição dos conceitos a respeito de um domínio de estudo.
			
			\item \textbf{Software:} os elementos de um sistema de software dizem a respeito diretamente dos elementos da UML. 
		\end{itemize}
	
	
	\section{Como chegamos à UML?}
	
		A partir da década de oitenta, com a popularização do paradigma orientado a objeto, muitas pessoas começaram a pensar sobre linguagens gráficas a fim de modelar problemas utilizando tal paradigma \cite{umlEssencial}.
		
		Existiam muitos profissionais que estudavam sobre o assunto na época, porém, havia algumas divergências sobre seus estudos. Somente com a aliança de \textbf{Booch} e \textbf{Rumbaugh} foi possível a publicação de um método padrão para a utilização e aperfeiçoamento das linguagens gráficas \cite{umlEssencial}.
		
		Somente após o ano de 1997, com a liderança de \textbf{Mary Loomis} e \textbf{Jim Odell}, começou-se a unificação de várias metodologias em favor de um padrão. Desta forma surgiu-se então a primeira versão do que conhecemos hoje como \textbf{UML} \cite{umlEssencial}.
		
		
	\section{Notações e Metamodelos}
	
		As notações na UML são basicamente a sintaxe gráfica da linguagem de modelagem, com objetivo de exemplificar, citamos a notação de diagrama de classes, que são representados os itens e conceitos, associação e multiplicidade.

		
		Existem algumas definições informais para explicar associação, multiplicidade ou classe. Então com intuito de atribuir significados formais para tais características, é possível indicar um \textbf{metamodelo}: diagrama que define conceitos da linguagem. Veja na Figura \ref{metamodelo1} um exemplo de metamodelo UML.
		
		\begin{figure}[H]
			\centering
			\begin{tikzpicture}[node distance=3cm, scale=0.5]
				\node (Caracteristica) [class, rectangle]{
					\textbf{Característica}
				};
			
				\node (Caracteristica Estrutural) [class, rectangle, below of=Caracteristica, left=50pt]{
					\textbf{Característica Estrutural}
				};   
			
				\node (Caracteristica Comportamental) [class, rectangle, below of=Caracteristica, right=50pt]{
					\textbf{Característica Comportamental}
				};
			
				\node (Parametro) [class, rectangle, below of=Caracteristica Comportamental]{
					\textbf{Parâmetro}
				};
			
				\draw[->, >=open triangle 90] (Caracteristica Estrutural) -| (Caracteristica);
				
				\draw[]  (Caracteristica Comportamental) -- (Caracteristica Estrutural);
			
				\draw[->, >=diamond] 
					(Parametro) -- node[pos=0.8, right, font=\fontsize{10}{10}] {$0..1$}  (Caracteristica Comportamental)
					(Parametro) -- node[pos=0.1, right, font=\fontsize{10}{10}] {$\textbf{*}$}  (Caracteristica Comportamental) 
					(Parametro) -- node[pos=0.12, left, font=\fontsize{10}{10}] {$\left\lbrace ordered \right\rbrace $}  (Caracteristica Comportamental);
			\end{tikzpicture}
			\caption{Uma pequena parte do metamodelo UML \\ Fonte: Martin, FOWLER,. UML Essencial. Grupo A, 2011.}
			\label{metamodelo1}
		\end{figure}
	
		\section{Diagramas UML}
		
		Existem 13 tipos de diagramas oficiais presentes na \textbf{UML 2}, que serão classificados ao decorrer da matéria \cite{umlEssencial}. Porém, como os tipos de diagramas não são totalmente rígidos, podemos utilizar um tipo de diagrama em outro. Veja na Figura \ref{tiposDiagramasUML} os tipos de diagramas UML.
		
		
		\begin{figure}[H]
			\centering
			\begin{tikzpicture}
				
				%NODES
				\node[UMLRectangle](diagrama) at (0,0){Diagrama};
				
				\node[UMLRectangle](estrutura) at (4,3){Diagrama de Estrutura};
				
				\node[UMLRectangle](comportamento) at (4,-4){\footnotesize Diagrama de Comportamento};
				
				\node[UMLRectangle](classes) at (9,6){Diagrama de Classes};
				
				\node[UMLRectangle](componentes) at (14,5){Diagrama de Componentes};					
				
				\node[UMLRectangle](estrutura compostas) at (9,3.35){Diagrama de Estrutura Compostas};
				
				\node[UMLRectangle](instalacoes) at (14,2.2){Diagrama de Instalações};
				
				\node[UMLRectangle](objetos) at (9, 1){ Diagrama de Objetos};
				
				\node[UMLRectangle](pacotes) at (14, 0){ Diagrama de Pacotes};
				
				\node[UMLRectangle](maquina de estado) at (9, -3.7){ Diagrama de Máquina de Estado};
				
				\node[UMLRectangle](atividades) at (9, -6){ Diagrama de Atividades};
				
				\node[UMLRectangle](caso de uso) at (9, -7.8){ Diagrama de Caso de Uso};
				
				\node[UMLRectangle](iteracoes) at (9, -9.5){ Diagrama de Iterações};
				
				\node[UMLRectangle](sequencia) at (14, -6){ Diagrama de Sequência};
				
				\node[UMLRectangle](comunicacao) at (14, -7.5){ Diagrama de Comunicação};
				
				\node[UMLRectangle](visao geral) at (14, -9.2){\footnotesize Diagrama de visão geral de iteração};
				
				\node[UMLRectangle](sincronizacao) at (14, -11.3){\footnotesize Diagrama de Sincronização};
				
				%EDGES CIMA ESTRUTURA
				
				\draw[->, >=open triangle 90] (estrutura) |- (diagrama);
				
				\draw[] (estrutura) -- (comportamento);
				
				\draw[->, >=open triangle 90] (classes) -| (estrutura);
				
				\draw[->, >=open triangle 90] (estrutura compostas) -- (estrutura);
				
				\draw[] (objetos) -|  (7, 2.4) (estrutura);
				
				\draw[->, >=open triangle 90] (componentes) -| (estrutura);
				
				\draw[] (instalacoes) -| (12, 4.35) (estrutura);
				
				\draw[] (pacotes) -| (12, 4.35) (estrutura);
				
				%EDGES CIMA COMPORTAMENTO
				
				\draw[->, >=open triangle 90] (maquina de estado) -- (comportamento);
				
				\draw[->, >=open triangle 90] (atividades) -| (comportamento);
				
				\draw[->, >=open triangle 90] (caso de uso) -| (comportamento);
				
				\draw[->, >=open triangle 90] (iteracoes) -| (comportamento);
				
				\draw[->, >=open triangle 90] (visao geral) -- (iteracoes);
				
				\draw[] (sequencia) -| (11.6, -10.14) (iteracoes);
				
				\draw[] (comunicacao) -| (11.6, -10.14) (iteracoes);
				
				\draw[] (sincronizacao) -| (11.6, -10.14) (iteracoes);
				
			\end{tikzpicture}
			\caption{Classificação dos tipos de diagrama da UML\\ Fonte: Martin, FOWLER,. UML Essencial. Grupo A, 2011.}
			\label{tiposDiagramasUML}
		\end{figure}
	
		A \textbf{UML} é uma linguagem que você pode esperar bastante regras prescritivas (estabelecidas por um comitê formador de padrões). Porém, como a mesma é muito complexa, acaba por ser frequentemente aberta à múltiplas interpretações.
		
		\section{Considerações Finais}
		
		Apesar de todas as qualidades da utilização de \textbf{UML}, um de seus defeitos é que o programador ao analisar o diagrama não irá saber exatamente como ficará o código final, somente ter ideias aproximadas.
		
		
		Desta forma fica evidente que não devemos ficar presos aos diagramas \textbf{UML}, ou seja, caso o mesmo não for suficiente para descrever nossas ideias, podemos recorrer a outras alternativas de diagramas e utiliza-los em conjunto.
		
		\nocite{*}
		\printbibliography
		
\end{document}
